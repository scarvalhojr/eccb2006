
\ignore{
For Border Length Minimization, the definition is again simple: We set
\begin{equation}
  d_{kl} := \frac{h(p_k, p_l)}{2},
\end{equation}
where $h(p_k, p_l)$ is the Hamming distance between the embeddings of probes
$p_k$ and $p_l$. We divide it by 2 for cosmetical reasons only: in
(\ref{eq:qap_def}), the conflicts between $p_k$ and $p_l$ appear twice (in
$d_{kl}$ and $d_{lk}$).

For conflict index minimization, the differences in the embeddings must be
weighted according to the number of nucleotides already used.
}



With these definitions in place, we show that the objective
function~(\ref{eq:qap_def}) is equal to the total conflict index
$\sum_i \mathcal{C}(i)$:



\ignore{
\subsubsection{Border Length Minimization}

The QAP formulation for the case of border length minimization is trivial. We set
%%
\begin{equation}
f_{ij} := \frac{h(p_i, p_j)}{2}
\end{equation}
%%
where $h(p_i, p_j)$ is the Hamming distance between the embeddings of probes $p_i$ and $p_j$. We divide it by since, in (\ref{eq:qap_def}), the conflicts between $p_i$ and $p_j$ appears twice (in $f_{ij}$ and $f_{ji}$). For the distance matrix, we set
%%
\begin{equation}
d_{ij} :=
        \left\{
                \begin{array}{ll}
                        1 & \mbox{if spots $i$ and $j$ are adjacent}, \\
                        0 & \mbox{otherwise}, \\
                \end{array}
        \right.
\end{equation}
%%
since only conflicts between adjacent spots are relevant for the border length. It is easy to verify that this formulation reflects the definition of border length.

\subsubsection{Conflict Index Minimization}

In case of conflict index minimization, the formulation is slightly more elaborate. Our goal is to design a microarray minimizing the sum of conflict indices over all spots $i$, that is
%%
\begin{equation}
\label{eq:ci_min}
\sum_{i} \mathcal{C}(i).
\end{equation}

From the point of view of a spot~$i$, there is a conflict at step~$k$ only when $i$ is masked and a close neighbor~$j$ is unmasked, in which case we say that there is an induced conflict of~$j$ onto~$i$, $\mathcal{C}_{j}(i)$, that can be derived from Equation (\ref{eq:conf_idx}) as
%%
\begin{equation}
\label{eq:induced_conf_idx}
\mathcal{C}_{j}(i) := \sum_{k=1}^{\mu} \omega(i,k) \cdot \delta(i,j,k).
\end{equation}

We can then rewrite (\ref{eq:conf_idx}) as
%%
\begin{equation}
\mathcal{C}(i) := \sum_{j} \mathcal{C}_{j}(i),
\end{equation}
%%
and our objective function (\ref{eq:ci_min}) turns into
%%
\begin{equation}
\label{eq:ci_min_qap}
\sum_{i} \sum_{j} \mathcal{C}_{j}(i),
\end{equation}
%%
where $j$ ranges over all neighbors that are at most three spots away from $i$.

Note the similarities between (\ref{eq:qap_def}) and (\ref{eq:ci_min_qap}). Now we need to set $f_{ij}$ and $d_{ij}$ in such a way that their multiplication results in $\mathcal{C}_{j}(i)$. The dependence of $\delta$ on $k$ is due to the fact that $\delta(i,j,k) = 0$ if spot~$j$ is masked at step $k$. It is thus possible to rewrite Equation (\ref{eq:induced_conf_idx}) as
%%
\begin{equation}
\label{eq:induced_conf_idx_2}
\mathcal{C}_{j}(i) := \left( \sum_{k=1}^{\mu} \omega(i,k) \cdot \phi(j,k) \right) \cdot (d(i,j))^{-2},
\end{equation}
%%
where
%%
\begin{equation}
\phi(j,k) :=
        \left\{
                \begin{array}{ll}
                        0 & \mbox{if spot $j$ is masked at step $k$}, \\
                        1 & \mbox{otherwise}, \\
                \end{array}
        \right.
\end{equation}
%%
and $d(i,j)$ is the Euclidean distance between spots~$i$ and~$j$ as used in (\ref{eq:dist_weight}).

Equation (\ref{eq:induced_conf_idx_2}) suggests how $f_{ij}$ and $d_{ij}$ can be set to produce $\mathcal{C}_j(i)$. The latter is trivial:
%%
\begin{equation}
d_{ij} :=
        \left\{
                \begin{array}{ll}
                        (d(i,j))^{-2} & \mbox{if spot $j$ is ``near'' spot $i$}, \\
                        0 & \mbox{otherwise}. \\
                \end{array}
        \right.
\end{equation}
%%
where ``near'' means that spot~$j$ is at most three cells away from~$i$. This definition also accounts for the difference in range of $j$ in (\ref{eq:qap_def}) and (\ref{eq:ci_min_qap}).

The only remaining problem is that $\mathcal{C}_j(i)$ is defined in terms of spots $i$ and $j$, whereas $f_{ij}$ must be defined in terms of probes $i$ and $j$, independently of which spots they are assigned to. However, the dependence of~$\omega$ and~$\phi$ on the spots is a mere convenience since the exact location of the spots is irrelevant. The embeddings of their probes is what matters. Hence, we set
%%
\begin{equation}
f_{ij} := \sum_{k=1}^{\mu} \omega'(i,k) \cdot \phi'(j,k),
\end{equation}
%%
where
%%
\begin{align}
\omega'(i,k) &:=
        \left\{
                \begin{array}{ll}
                        c \cdot \exp{\left(\theta \cdot \lambda'(i,k)\right)} &
                            \mbox{if embedding of $i$} \\
                          & \mbox{is masked at step $k$}, \\
                        0 & \mbox{otherwise}, \\
                \end{array}
        \right. \\
\lambda'(i,k) &:= 1 + \min(b'_{i,k},\ell'_{i} - b'_{i,k}), \\
\phi'(j,k) &:=
        \left\{
                \begin{array}{ll}
                        0 & \mbox{if embedding of $j$ is masked at step $k$} \\
                        1 & \mbox{otherwise}, \\
                \end{array}
        \right.
\end{align}
%%
$c$ and $\theta$ are constants, $\ell'_i$ is the length of probe $i$, and $b'_{i,k}$ denotes the number of nucleotides of probe~$i$ synthesized up to and including step~$k$.

}


%%% Local Variables: 
%%% mode: latex
%%% TeX-master: "eccb2006"
%%% End: 
